\documentclass[11pt,]{article}
\usepackage[left=1in,top=1in,right=1in,bottom=1in]{geometry}
\newcommand*{\authorfont}{\fontfamily{phv}\selectfont}
\usepackage[sc, osf]{mathpazo}


  \usepackage[T1]{fontenc}
  \usepackage[utf8]{inputenc}



\usepackage{abstract}
\renewcommand{\abstractname}{}    % clear the title
\renewcommand{\absnamepos}{empty} % originally center

\renewenvironment{abstract}
 {{%
    \setlength{\leftmargin}{0mm}
    \setlength{\rightmargin}{\leftmargin}%
  }%
  \relax}
 {\endlist}

\makeatletter
\def\@maketitle{%
  \newpage
%  \null
%  \vskip 2em%
%  \begin{center}%
  \let \footnote \thanks
    {\fontsize{18}{20}\selectfont\raggedright  \setlength{\parindent}{0pt} \@title \par}%
}
%\fi
\makeatother




\setcounter{secnumdepth}{0}

\usepackage{longtable,booktabs}

\usepackage{graphicx,grffile}
\makeatletter
\def\maxwidth{\ifdim\Gin@nat@width>\linewidth\linewidth\else\Gin@nat@width\fi}
\def\maxheight{\ifdim\Gin@nat@height>\textheight\textheight\else\Gin@nat@height\fi}
\makeatother
% Scale images if necessary, so that they will not overflow the page
% margins by default, and it is still possible to overwrite the defaults
% using explicit options in \includegraphics[width, height, ...]{}
\setkeys{Gin}{width=\maxwidth,height=\maxheight,keepaspectratio}

\title{External Territorial Threats and Tolerance of Corruption: A
Private/Government Distinction \thanks{Replication files are available on the author's Github account
(\href{https://github.com/svmiller}{github.com/svmiller}).
\textbf{Corresponding author}:
\href{mailto:svmille@clemson.edu}{\nolinkurl{svmille@clemson.edu}}.}  }



\author{\Large Steven V. Miller\vspace{0.05in} \newline\normalsize\emph{Clemson University}  }


\date{}

\usepackage{titlesec}

\titleformat*{\section}{\normalsize\bfseries}
\titleformat*{\subsection}{\normalsize\itshape}
\titleformat*{\subsubsection}{\normalsize\itshape}
\titleformat*{\paragraph}{\normalsize\itshape}
\titleformat*{\subparagraph}{\normalsize\itshape}





\newtheorem{hypothesis}{Hypothesis}
\usepackage{setspace}

\makeatletter
\@ifpackageloaded{hyperref}{}{%
\ifxetex
  \PassOptionsToPackage{hyphens}{url}\usepackage[setpagesize=false, % page size defined by xetex
              unicode=false, % unicode breaks when used with xetex
              xetex]{hyperref}
\else
  \PassOptionsToPackage{hyphens}{url}\usepackage[unicode=true]{hyperref}
\fi
}

\@ifpackageloaded{color}{
    \PassOptionsToPackage{usenames,dvipsnames}{color}
}{%
    \usepackage[usenames,dvipsnames]{color}
}
\makeatother
\hypersetup{breaklinks=true,
            bookmarks=true,
            pdfauthor={Steven V. Miller (Clemson University)},
             pdfkeywords = {territorial conflict, government corruption, private corruption,
tolerance of corruption},  
            pdftitle={External Territorial Threats and Tolerance of Corruption: A
Private/Government Distinction},
            colorlinks=true,
            citecolor=blue,
            urlcolor=blue,
            linkcolor=magenta,
            pdfborder={0 0 0}}
\urlstyle{same}  % don't use monospace font for urls

% set default figure placement to htbp
\makeatletter
\def\fps@figure{htbp}
\makeatother



% add tightlist ----------
\providecommand{\tightlist}{%
\setlength{\itemsep}{0pt}\setlength{\parskip}{0pt}}

\begin{document}
	
% \pagenumbering{arabic}% resets `page` counter to 1 
%
% \maketitle

{% \usefont{T1}{pnc}{m}{n}
\setlength{\parindent}{0pt}
\thispagestyle{plain}
{\fontsize{18}{20}\selectfont\raggedright 
\maketitle  % title \par  

}

{
   \vskip 13.5pt\relax \normalsize\fontsize{11}{12} 
\textbf{\authorfont Steven V. Miller} \hskip 15pt \emph{\small Clemson University}   

}

}








\begin{abstract}

    \hbox{\vrule height .2pt width 39.14pc}

    \vskip 8.5pt % \small 

\noindent What makes individuals tolerate government corruption? Can citizens
tolerate government corruption but be intolerant of corrupt behavior in
society? I argue not all attitudes toward corruption are the same.
External territorial threats elicit a tolerance of government corruption
since citizens allow for government corruption when they are concerned
for their security. However, citizens become intolerant of corruption in
society because they view this as maximizing individual welfare at the
expense of the common good (i.e.~security). Using data from three unique
cross-national surveys, I find that citizens under territorial threat
are less likely to think corruption is an important problem, are more
likely to tolerate government corruption, but are less likely to
tolerate corruption by private citizens.


\vskip 8.5pt \noindent \emph{Keywords}: territorial conflict, government corruption, private corruption,
tolerance of corruption \par

    \hbox{\vrule height .2pt width 39.14pc}



\end{abstract}


\vskip -8.5pt


\noindent  \section{Introduction}\label{introduction}

Government corruption, understood as the misuse of public office for
private gain, seems inimical to the interests of society. The extent to
which government actors, especially in nascent democracies, steal state
resources or allocate resources to those who pay outside legal channels
represents the extent to which democracy's citizens question the regime
as a legitimate form of government. The spate of government corruption
scandals, just in Latin America in the 1990s, underscores the importance
of corruption as policy issue. In this region alone, nine heads of state
were forced from office in that decade for using state resources to
subsidize a lavish private life, reward cronies, or otherwise operate
outside legal restrictions. When public officials and heads of state use
their position for personal gain, trust in the democratic experiment
falters.

Research into government corruption, especially research focused on
corruption scandals, tends to build in strong assumptions that the
perception of corruption is equivalent to the intolerance of corruption.
It is intuitive that these are related. After all, perceiving the
corruption of Carlos Andrés Pérez in November 1992, who embezzled 250
million bolivars, led to his removal from the Venezuelan presidency in
May 1993. It is convenient and easy to assume that behavior perceived as
corrupt is behavior that warrants punishment.

However, many corrupt governments survive in office. Silvio Berlusconi
was the longest-serving post-war Prime Minister of Italy despite being
the subject of a multitude of judicial inquiries and accusations of
criminal malfeasance for almost the entirety of his first eleven
non-consecutive years in office. Nursultan Nazarbayev, President of
Kazakhstan, is one of the most notorious kleptocrats in Central Asia
(Kramer 2005), but his popular approval is as high as 90 percent in
polls conducted by foreign firms (Lillis 2010). Can individuals perceive
government corruption but tolerate its presence? In addition, can
individuals extend this liberty to government actors to behave corruptly
but condemn corruption from non-government actors in society?

I propose an answer to both the issue of tolerance of government
corruption and tolerance of private (non-government) corruption.
Building from territorial conflict literature, I argue that external
threats to the state can elicit this individual-level tolerance of
government corruption. Disputes over territory in the international
system constitute a threat to the individual citizen, who looks to the
state leader's government to provide security. When individuals value
the government providing for their security, they are willing to reduce
oversight on government behavior elsewhere. This allows state leaders
and their governments to behave corruptly, which the citizen permits.

However, the same conflict process that draws individuals toward their
government makes them intolerant of their compatriots who deviate from
group norms. Deviating from non-sanctioned behavior constitutes a sign
of weakness that individuals cannot tolerate under threat. Individuals
in society who take bribes in the course of their duties are seen as
acting for individual gain at a critical time when the pursuit of the
common good is necessary. Territorial threat leads individuals to
tolerate government corruption but leads them to be intolerant of
corruption in society. My analyses of cross-national survey data from
Globalbarometer, Latinobarómetro, and World Values Survey affirm these
expectations.

The paper is divided into the following sections. First, I review the
literature on corruption, noting its policy concerns and the conflation
of corruption \emph{perception} as corruption \emph{intolerance}.
Thereafter, I develop a theory that links territorial threat to
individual-level tolerance of government corruption and intolerance of
what I term ``private corruption'' (i.e.~corrupt behavior not involving
the national government). Thereafter, I test my hypotheses with three
different cross-national survey data sets, finding support for my
arguments. My conclusions follow.

\section{Government Corruption: Its Perception and
Tolerance}\label{government-corruption-its-perception-and-tolerance}

Is corruption, defined as the use of public office for personal gain
(Williams 1999), even a bad thing for the administration and prosperity
of a country? Political scientists sang the praises of corruption in
politics (Huntington 1968; Beck and Maher 1986; Leys 1989) for many
decades while economists decried that corruption of any kind distorts
economic outputs (Krueger 1974; Mauro 1995). Only after the Cold War did
political scientists begin to see corruption as something worse than a
``necessary evil.'' The more corrosive elements of government corruption
became apparent when the need to support corrupt governments that were
anti-communist disappeared. Corruption erodes trust in the government
and between people as well (Mishler and Rose 2001; Rose-Ackerman 2001;
Chang and Chu 2006). Citizens who do not trust government institutions
amid widespread corruption become corrupt themselves (Della Porta 2000;
Morris and Klesner 2007).

All told, widespread government corruption is trouble for the legitimacy
of government institutions in nascent democracies. In the fledgling
democracies of Central Europe and Eastern Europe, corruption decreases
support for the democratic project and increases support for a return to
authoritarianism (Rose, Mishler, and Haerpfer 1998). In Latin America,
widespread government corruption erodes legitimacy for the democratic
political system (Seligson 2002; Canache and Allison 2005). This makes
government corruption in developing countries a normative concern for
those interested in democratic consolidation.

This new concern for the pernicious effects of corruption in developing
countries had implications for policymakers (Manzetti and Wilson 2007,
950--51). The end of the Cold War and dissolution of the Soviet Union
removed any normative justification for Western powers of promoting
corrupt behavior in third-world governments provided those corrupt
governments were anti-communist. The abandonment of centrally-planned
economies in post-Soviet republics and the simultaneous discontinuation
of import substitution industrialization in South America afforded new
opportunities for the World Bank and International Monetary Fund to
promote neoliberalism across the globe. However, the promotion of open
markets, trade liberalization, and deregulation through the use of
capital investments came with the concern that recipient governments
would put these investments to good use. These institutions would no
longer condone diversion of foreign investments into personal projects
or bank accounts through conventional government corruption. Toward that
end, these institutions developed and supported new research agendas on
corruption that continue to be policy-relevant, including how to measure
the extent of corruption in a country when the act, by its nature, is
sub rosa.

The issue of measurement is important, but measuring the extent of
corruption in a country is difficult and is prone to bias. One of the
earliest indicators for the extent of corruption looked at the number of
arrests. This approach argues that that areas with high levels of
corruption will also have higher arrests on the charge (Schlesinger and
Meier 2002, for a contemporary example). However, this criminology
approach to the measurement of corruption also measures \emph{vigilance}
against corruption and not the \emph{extent} of corruption itself.
Corrupt regimes may appear spotless for not having enforcement devices
in place whereas honest regimes may appear corrupt for vigilance against
minor infractions. To overcome this limitation, corruption research
measures the extent of corruption by reference to the perception of
corruption. This is at the heart of Transparency International's
Corruption Perceptions Index (CPI), which, since 1995, has annually
measured the extent of corruption in countries across the globe using
expert assessments and opinion surveys.

This approach, also used by the Political Risk Services firm in its
International Country Risk Guide, is straightforward. If anyone would
know about the true extent of corruption in a country, it would be risk
analysts, international and country business elites, and country
experts. Even the World Bank's ``Control of Corruption'' indicator draws
upon these expert opinion surveys, using an unobserved components model
to create a weighted average that serves an estimate of the ``true''
level of corruption in a country. While this approach is not without its
own set of limitations (e.g. Knack 2007; Olken 2009), the measurement of
corruption as the perception of corruption remains the most common and
accessible approach for academics and policymakers.

An unintended side effect of the perception approach to the measurement
of corruption has been the conflation of corruption perception and
corruption \emph{tolerance} as equivalent. For example, Heidenheimer
(2002) seems to define the perception of corruption as existing when
individuals observe behavior by public officials that they believe
should be punished. Heidenheimer describes a potential corrupt act that
is decried as corrupt by a single citizen, who also feels the public
official responsible for the act should be punished. If, however, 99\%
of the citizen's community disagree with the conscious citizen that the
act should be punished, then the act is not considered corrupt by the
community's standards (Heidenheimer 2002, 152). An act not perceived as
corrupt by the community becomes acceptable per this logic. If the act
is perceived as corrupt, such as then-U.S. Senator Thomas Dodd's 1967
use of campaign funds for personal projects in Heidenheimer's example,
then it also warrants punishment (such as Dodd's censure that year in
the Senate). The perception approach assumes it is self-evident that
corruption perception and corruption tolerance are equivalent. Though
this is intuitive, these are separable concepts. An individual does not
have to be intolerant of behavior s/he observes as corrupt.

What leads individuals to tolerate government corruption? Pani (2011)
argues that government corruption leads to a decrease in public
expenditure and the tax rate, which even non-corrupt and honest citizens
in developed democracies may find acceptable if it mirrors their
preference for reduced government spending. Manzetti and Wilson (2007)
find that support for corrupt governments---i.e.~tolerance of government
corruption---is more often observed in countries where government
institutions are weak and patron-client relationships are strong. Thus,
citizens see material value in its government behaving corruptly
provided they are in the private goods network. Chang and Kerr (2009)
build on this idea of clientelism conditioning perceptions and tolerance
of government corruption. Articulating an insider-outsider theory of
perceptions and tolerance of corruption, the authors argue that
patronage network insiders are more likely to accurately perceive
corruption. They are also more likely to tolerate it. Beyond that, not
much exists separating perception of government corruption from
tolerance of government corruption. Tolerance of corruption has been a
neglected topic in a literature focused on the issue of perception
(Manzetti and Wilson 2007, 950).

This disparity about what we know of tolerance of government corruption
becomes more pronounced in available research that uses the standard
question of the justifiability of bribe-taking. This survey question is
available in many data sets and is a staple in the World Values Survey.
It asks respondents to say, on a ten-point scale, how justifiable it
would be if ``someone accept(ed) a bribe in the course of their
duties.'' No additional context about the ``someone'' is provided in
this question, making it more a question of what is an acceptable
\emph{social} behavior (Katza, Santmana, and Loneroa 1994; Donga,
Dulleckb, and Torgler 2012) rather than a question of what is an
acceptable \emph{political} behavior. The respondent's answer to the
justifiability of bribe-taking is likely conditioned on the belief that
the questioner is talking about private-private corruption, rather than
private-public corruption (Argandona 2003).

Private-private corruption, which I label as ``private corruption'' in
this manuscript, is altogether different from standard ``government
corruption'' involving a government official's misuse of public funds
for personal gain (Goel, Budak, and Rajh 2015). Corruption in the
private case does not involve any public official. The 2003 U.S.
financial scandal involving Jack Grubman is a notable example of private
corruption. Jack Grubman, then a stock analyst for Citigroup, upgraded
AT\&T's stock from ``neutral'' to ``buy'' in a multi-tiered illicit
exchange, where, in return, AT\&T and Citigroup's CEOs helped place
Grubman's twin daughters in a prestigious and selective New York
preschool. A respondent to the ten-point bribe-taking justifiability
question may be critical of that form of bribe-taking in the private
sector, but could be ambivalent, or tolerant, of government corruption.
Much like individuals can perceive corruption in the government and
tolerate it, citizens can be tolerant of government corruption but
intolerant of private corruption.

When do citizens tolerate corruption by the government? In addition,
when do citizens tolerate government corruption while condemning
corruption at the societal level? In the next section, I propose a
theory linking external territorial threats with this asymmetry in
corruption tolerance. I argue that citizens in states under territorial
threat are likely to tolerate government corruption. This does not
extend into the realm of private corruption. Citizens are unlikely to
tolerate private corruption amongst their compatriots in times of threat
to territory.

\section{Territorial Threat and Varying Tolerance for
Corruption}\label{territorial-threat-and-varying-tolerance-for-corruption}

My argument linking territorial disputes to tolerance of corruption at
the individual-level begins with the well-established salient nature of
this particular type of external threat. The salience of territory
affects both the general public and the state elites. The general public
has an emotional investment in territory, creating identities around the
homeland (Tir 2010; Gibler 2012). State elites, in particular, are
``soft-wired'' toward aggressive overtures against external rivals over
the issue of territory, having learned through time and history that
violence is the best means to securing what is essential to survival and
prosperity of a society (Vasquez 2009). In modern times, this assumes
the form of militarization and mobilization of the armed forces in times
of a territorial crisis. The inclination to respond to territorial
threats with military overtures makes territorial disputes a first
``step to war'' (Senese and Vasquez 2003).

This mobilization of the military has important implications beyond the
maintenance of an adequate balance of power with the territorial rival.
An abrupt mobilization of the military leads to different levels of
concern among actors in society. Though a myriad of possible
socio-economic groups exist within a given state, two broad groups---the
wealthy elites and general public---are most relevant. The elites
consist of two subgroups. First, the civilian elites are the small
subset of society whose assets constitute the nation's overall wealth.
Second, the military can often be conceptualized as a subset of the
elite, given the military's proclivity to be bribed into alignment with
the elite against the general public. By contrast, the general public
differs from the wealthy elites in both assets and overall proportion of
a population. The abstraction of society into these two groups is a
useful foundation for influential arguments on democratization,
redistribution, and state development (Boix 2003; Acemoglu and Robinson
2006).

Understanding society as consisting of two relevant groups is important
because the wealthy elites and general public have different reactions
to increases in defense spending and military mobilization. Increased
defense spending by the state leader benefits the wealthy elites. The
military, as a subset of the wealthy elites, benefits from defense
spending since it is a direct subsidization of the military's
livelihood. The civilian elites benefit from increased defense spending
since a better equipped and mobilized army serves as insurance against
the demands from the general public for redistributive policies to
ameliorate the inequality between rich and poor. Defense spending makes
repression, via the military, cheaper for the civilian elites.

The general public has a different reaction to defense spending, all
else equal. Resource scarcity dictates that increased defense spending
comes at the expense of social spending aimed at reducing the inequality
between rich and poor. This does not discount that national defense is a
public good that the general public enjoys, but increased spending on
defense depresses consumption spending on social programs from which the
general public benefits. Spending on the military above the bare minimum
necessary to provide for national defense is suboptimal for the general
public and an issue for concern.

This becomes a monitoring problem for the general public. Concerned with
what resources are being diverted from social spending, the general
public prefers to know what is being allocated in order to ensure an
optimal division of state resources between social spending and other
government priorities. However, a rapid mobilization in response to
territorial threat is discretionary. It requires decisions made by the
state leader about the best possible deployment of the military and
additional state resources. A threat to territory does not permit time
for bureaucratic process or red tape that would be seen as hindrances to
territorial defense. This also means the general public is unaware if
the decisions made by the state leader about the diversion of state
resources from social spending, including a diversion of funds toward
government corruption, are for their detriment.

All else equal, the general public could try to coerce the allocation by
the state leader closer to its preference through elections or the
threat of rebellion. However, territorial threat, as a unique kind of
threat external to the state, leads to compliance from the general
public. This is the familiary ``rally effect'' where in-group/out-group
distinctions between states magnifies in an international crisis to the
benefit of the state leader. In democracies, this ``rally effect'' leads
to a groundswell of support for the state leader that makes it untenable
for opposition parties to challenge the state leader's policies during a
crisis (Miller 2017b). Even in autocratic states, the general public,
members of whom are likely to rest outside the state leader's private
goods network (Bueno de Mesquita et al. 2003), may rally around the
state leader when the expected costs of direct combat over the territory
proximate to the general public outweigh the expected benefits of
foreign imposed regime change following a defeat in conflict (Gibler
2010, 523--24). Thus, the citizen prefers to slacken oversight of the
state leader's spending decisions when concerns for security are
paramount (see also: Bueno de Mesquita 2007). The presence of a threat
to territory induces a ``rally effect'' that makes it unlikely the
general public will challenge the discretionary decisions made by the
state leader. This allows the state leader to engage in government
corruption that the general public would tolerate in times of
territorial threat. This argument implies two hypotheses.

\begin{hypothesis}
\begin{minipage}[t]{5.1 in}
Citizens in states under territorial threat are unlikely to say corruption is the most important problem in their country.
\end{minipage}
\end{hypothesis}

\begin{hypothesis}
\begin{minipage}[t]{5.1 in}
Citizens in states under territorial threat express a tolerance of government corruption.
\end{minipage}
\end{hypothesis}

This tolerance of government corruption does not extend to the rest of
society. Extant territorial threat scholarship underscores that the
effect of territorial threat on attitudes toward state leadership is not
the same as attitudes toward other individuals and society in general.
This may be a function of the same rally effect that follows territorial
threat. The rally effect draws individuals toward state leadership,
through which individuals hope to achieve unity in the face of the
threat. However, this is not a friction-less process. Hutchison and
Gibler (2007) make this point when they argue that territorial threat is
an in-group socializing mechanism. A salient form of threat with a
higher probability of war onset than other foreign policy issues,
citizens are justified in fearing the onset of conflict on the homeland.
In response, individuals cleave together as a show of resolve against
the source of the territorial threat. Using World Values Survey data,
the authors argue that allowing self-identified least-liked groups the
right to demonstrate or the right to run for office is discouraged by
citizens under conditions of external territorial threat. Allowing
self-identified least-liked groups these political liberties would be
seen as introducing weaknesses into the political system at a critical
time when strength is necessary.

A general intolerance for disreputable behavior among peers extends to
behavior considered to be corrupt, like taking a bribe. Under conditions
of territorial threat, citizens can no longer express a willingness to
``put up with'' a behavior from their compatriots that is considered to
be objectionable. This echoes the classic definition of intolerance
offered by Sullivan, Marcus, and Pierson (1979). The implication here is
two-fold. Bribe-taking among private actors in society is not just
considered wrong; it is also resisted. As Hutchison (2011) argues,
conditions of territorial threat allow state elites to mobilize public
opinion toward the position of the state's elites, leading to a broader
societal effect where individuals are mobilized by their governments to
participate among government-sanctioned lines. Individual departures
from what the government endorses, such as a shopkeeper accepting a
bribe for a greater quantity of a good that may be rationed by the
government, is seen as participating in behavior not condoned by the
government. Citizens interpret such an activity as maximizing individual
welfare at the expense of working together for the common good.

Though individuals will condone government corruption and loosen
oversight of government activities when they value their security, they
are unlikely to afford this same liberty to their fellow citizens. The
same process that leads individuals to rally toward their government
under threat, while being intolerant of deviations from group norms in
society, also leads individuals to permit opportunities for the
government they will not permit for their fellow citizens in society.
Under conditions of territorial threat, citizens tolerate government
corruption but do not tolerate private corruption. This suggests a third
hypothesis.

\begin{hypothesis}
\begin{minipage}[t]{5.1 in}
Citizens in states under territorial threat express an intolerance of private corruption.
\end{minipage}
\end{hypothesis}

In the next section, I outline a test of this argument, drawing off a
unique data source that will permit me the opportunity to test both
claims.

\section{Research Design}\label{research-design}

I test my argument drawing data from three cross-national survey data
sets (Globalbarometer, Latinobarómetro, World Values Survey) to estimate
an individual's tolerance or permissive attitudes toward government and
private corruption. The first data source, Globalbarometer, represents
the most careful and systematic comparative survey of attitudes and
values toward politics. Polling 55 countries between 2003 and 2006, the
Globalbarometer project eschewed the North American and European focus
of the World Values Survey in favor of a focus on Africa, Asia, Latin
America, and the Middle East.

I also use survey data from the 2002 wave of Latinobarómetro. The 2002
wave of Latinobarómetro is useful, in particular, because the United
Nations Development Programme (UNDP) and its regional offices in Latin
America and the Caribbean worked closely with the Chilean cross-national
polling firm in this survey to gauge individual-level attitudes toward
authoritarian rule and corruption in seventeen countries in Central
America and South America (Caputo 2005). The questions about support for
democracy and attitudes toward corruption in this particular survey wave
are much more comprehensive than other cross-national survey data sets.

Finally, I draw data from the fourth and fifth waves of the World Values
Survey project, which polled countries from 1999 to 2008. Whereas
previous waves of the World Values Survey project oversampled OECD
countries in North America and Western Europe, later waves have included
countries from across the globe. The considerable variation in economic
development, institutional development, and regional threat environment
for these countries makes newer waves of World Values Survey a more
representative sample of the rest of the globe than previous waves.

\subsection{Dependent Variables}\label{dependent-variables}

I derive four dependent variables from these three data sets to analyze
in the following section. The first dependent variable comes from a
Globalbarometer question that asks the respondent to state what is the
country's most important national problem using a comprehensive list
provided by Globalbarometer. Globalbarometer featured 63 options, which
included security problems like terrorism, health issues like AIDS and
other endemic diseases, sociodemographic issues like an aging
population, and various problems associated with a malfunctioning
government or economy. Corruption was one of the listed problems, which
eight percent of respondents selected in the survey. I recode this
variable to equal 1 if the respondent believes corruption is the most
important problem and 0 if the respondent does not believe corruption is
the most important problem. If my hypotheses are correct, respondents
are unlikely to believe corruption is a country's most important problem
under conditions of territorial threat.

The next two dependent variables come from the 2002 wave of
Latinobarómetro data, which is unique for its comprehensive questions on
attitudes toward corruption. One particular question from this wave asks
the citizen to say whether a certain degree of corruption by the
government is okay as long as the problems of the country are being
solved. The individual in the survey then responds that they strongly
agree, somewhat agree, somewhat disagree, or strongly disagree that a
level of corruption in the government is permissible, provided national
problems facing the government and its citizens are being addressed.
Unlike the standard ten-point justifiability of bribe-taking question,
this question primes the respondent to think of government corruption
and not private corruption. I condense this to a dichotomous measure
that equals a 1 if the respondent thinks a certain degree of corruption
by the government is okay as long as the problems of the country are
being solved.

The second dependent variable is the common ten-point justifiability of
bribe-taking question, which is also available in the Latinobarómetro
data set. If my theory is correct, a citizen is more likely to tolerate
corruption by the government under conditions of territorial threat but
will be intolerant of bribe-taking among the individual's peers in
society. This is because the bribe-taking question is more of a question
about acceptable social behavior rather than a metric on tolerance of
government corruption. I follow a common way of dealing with this
variable's non-normal distribution (see: Swamy et al. 2001) and condense
this variable to a dichotomous measure. A value of 0 indicates the
respondent believes it is never okay to take a bribe. A value of 1
indicates some tolerance in the survey respondent of taking a bribe.

Descriptive statistics support this distinction at first glance. The
mean of the dichotomous bribe-taking variable is .166, indicating that
84\% of respondents believe it is never justifiable to accept a bribe.
However, the percentage of those answering either ``somewhat agree'' or
``strongly agree'' to the dependent variable on tolerance of government
corruption is 60\%. The two measures have a Pearson's R of only -.05,
which suggests the two dependent variables are indeed analytically
distinct and refer to different attitudes towards different forms of
corruption.

Finally, I take the ten-point justifiability of bribe-taking question
from the World Values Survey and recode it as a dependent variable
similar to the analyses shown using Latinobarómetro data. Table 1
provides variable definitions, summary statistics, and data sources for
the four dependent variables in my analysis.

\begin{longtable}[]{@{}lllc@{}}
\caption{Variable Definitions, Summary Statistics and Data Sources for
the Four Dependent Variables}\tabularnewline
\toprule
\begin{minipage}[b]{0.10\columnwidth}\raggedright\strut
\textbf{Model}\strut
\end{minipage} & \begin{minipage}[b]{0.14\columnwidth}\raggedright\strut
\textbf{Variable}\strut
\end{minipage} & \begin{minipage}[b]{0.37\columnwidth}\raggedright\strut
\textbf{Description {[}N; mean; std. dev.{]}}\strut
\end{minipage} & \begin{minipage}[b]{0.21\columnwidth}\centering\strut
\textbf{Source}\strut
\end{minipage}\tabularnewline
\midrule
\endfirsthead
\toprule
\begin{minipage}[b]{0.10\columnwidth}\raggedright\strut
\textbf{Model}\strut
\end{minipage} & \begin{minipage}[b]{0.14\columnwidth}\raggedright\strut
\textbf{Variable}\strut
\end{minipage} & \begin{minipage}[b]{0.37\columnwidth}\raggedright\strut
\textbf{Description {[}N; mean; std. dev.{]}}\strut
\end{minipage} & \begin{minipage}[b]{0.21\columnwidth}\centering\strut
\textbf{Source}\strut
\end{minipage}\tabularnewline
\midrule
\endhead
\begin{minipage}[t]{0.10\columnwidth}\raggedright\strut
Model 1\strut
\end{minipage} & \begin{minipage}[t]{0.14\columnwidth}\raggedright\strut
\emph{Corruption is Most Important Problem}\strut
\end{minipage} & \begin{minipage}[t]{0.37\columnwidth}\raggedright\strut
Codes a 1 if respondent thinks corruption is the country's most
important problem, 0 if some other listed problem is most important
{[}48,040; .075; .263{]}.\strut
\end{minipage} & \begin{minipage}[t]{0.21\columnwidth}\centering\strut
Globalbarometer (GB)\strut
\end{minipage}\tabularnewline
\begin{minipage}[t]{0.10\columnwidth}\raggedright\strut
Model 2\strut
\end{minipage} & \begin{minipage}[t]{0.14\columnwidth}\raggedright\strut
\emph{Tolerance of Government Corruption}\strut
\end{minipage} & \begin{minipage}[t]{0.37\columnwidth}\raggedright\strut
Codes a 1 if respondent thinks a certain degree of corruption by the
government is okay as long as the problems of the country are being
solved, 0 if not okay. {[}16,679; .166; .372{]}\strut
\end{minipage} & \begin{minipage}[t]{0.21\columnwidth}\centering\strut
Latinobarómetro (LB)\strut
\end{minipage}\tabularnewline
\begin{minipage}[t]{0.10\columnwidth}\raggedright\strut
Model 3\strut
\end{minipage} & \begin{minipage}[t]{0.14\columnwidth}\raggedright\strut
\emph{Tolerance of Private Corruption}\strut
\end{minipage} & \begin{minipage}[t]{0.37\columnwidth}\raggedright\strut
Codes a 1 if respondent thinks someone accepting a bribe is in any way
justifiable, 0 if it's never justifiable. {[}16,679; .591; .492{]}\strut
\end{minipage} & \begin{minipage}[t]{0.21\columnwidth}\centering\strut
Latinobarómetro (LB)\strut
\end{minipage}\tabularnewline
\begin{minipage}[t]{0.10\columnwidth}\raggedright\strut
Model 4\strut
\end{minipage} & \begin{minipage}[t]{0.14\columnwidth}\raggedright\strut
\emph{Tolerance of Private Corruption}\strut
\end{minipage} & \begin{minipage}[t]{0.37\columnwidth}\raggedright\strut
Codes a 1 if respondent thinks someone accepting a bribe is in any way
justifiable, 0 if it's never justifiable. {[}98,246; .238, .426
{]}\strut
\end{minipage} & \begin{minipage}[t]{0.21\columnwidth}\centering\strut
World Values Survey (WVS)\strut
\end{minipage}\tabularnewline
\bottomrule
\end{longtable}

\subsection{Measuring Territorial
Threat}\label{measuring-territorial-threat}

I get my measure of territorial threat is taken from the Correlates of
War Militarized Interstate Dispute (COW-MID) data set (Palmer et al.
2015; Gibler, Miller, and Little 2016). Defined as either a threat,
display or use of force, militarized interstate disputes provide a great
measure for the presence of external threat in a country. Militarized
interstate disputes (MIDs) are diverse, encompassing a variety of
actions and stem from a variety of issues.

For every country in each of these three data sources, I look through
the MID history in the five years prior to the survey year. I
disaggregate the type of MID into four categories based on whether the
MID was over the distribution of territory or some other issue and
whether the country initiated the MID or was targeted by another state.
To assess whether the dispute was over territory, I consult the the MID
3.0 and MID 4.0 narratives file maintained by the Correlates of War
project as well as the \emph{revtype} variables provided in the data
set. In all cases, whether the dispute was over territory or some other
issue was clear.

After separating the MIDs by type (non-territorial and territorial), I
further delineate MIDs in which the state in question was the clear
initiator or MIDs in which the state in question was the clear target.
As more recent scholarship on territorial conflict suggests, the
conflict process for initiators is not equivalent to the conflict
process for targets (Tir 2010; Miller 2013). Tolerance of government
corruption and intolerance of private corruption are expected for those
citizens who are in states that are the targets of territorial threat.
Determining initiators and targets of MIDs was done through consulting
the \emph{revstate} and \emph{sidea} variables maintained by the COW-MID
data set and reading through the accompanying narratives maintained by
the Correlates of War project. Following Hutchison and Gibler (2007), I
code a case where there was no clear initiator as a case where each side
involved was targeted.

The predictor variables drawn from the MID data represent the presence
or absence of a particular MID type in the five years prior to the
survey year.

\subsection{Control Variables}\label{control-variables}

I employ several control variables from a review of the corruption
literature. I took care to isolate variables that explain
individual-level perception or tolerance of corruption, rather than
variables that explain the observation of corruption at the societal
level without clear implications for the dependent variables of
interest. Table 2 provides variable definitions and data sources for
these variables at both the micro-level and macro-level. The appendix
contains descriptive statistics (i.e.~means, standard deviations) for
each of these variables by cross-national survey data set and analysis.

\begin{longtable}[]{@{}llc@{}}
\caption{Variable Definitions and Data Sources}\tabularnewline
\toprule
\begin{minipage}[b]{0.19\columnwidth}\raggedright\strut
\textbf{Variable}\strut
\end{minipage} & \begin{minipage}[b]{0.46\columnwidth}\raggedright\strut
\textbf{Description}\strut
\end{minipage} & \begin{minipage}[b]{0.22\columnwidth}\centering\strut
\textbf{Source}\strut
\end{minipage}\tabularnewline
\midrule
\endfirsthead
\toprule
\begin{minipage}[b]{0.19\columnwidth}\raggedright\strut
\textbf{Variable}\strut
\end{minipage} & \begin{minipage}[b]{0.46\columnwidth}\raggedright\strut
\textbf{Description}\strut
\end{minipage} & \begin{minipage}[b]{0.22\columnwidth}\centering\strut
\textbf{Source}\strut
\end{minipage}\tabularnewline
\midrule
\endhead
\begin{minipage}[t]{0.19\columnwidth}\raggedright\strut
\emph{Age}\strut
\end{minipage} & \begin{minipage}[t]{0.46\columnwidth}\raggedright\strut
Age of respondent in years\strut
\end{minipage} & \begin{minipage}[t]{0.22\columnwidth}\centering\strut
GB, LB, WVS\strut
\end{minipage}\tabularnewline
\begin{minipage}[t]{0.19\columnwidth}\raggedright\strut
\emph{Female}\strut
\end{minipage} & \begin{minipage}[t]{0.46\columnwidth}\raggedright\strut
Dummy for if respondent self-identifies as female\strut
\end{minipage} & \begin{minipage}[t]{0.22\columnwidth}\centering\strut
GB, LB, WVS\strut
\end{minipage}\tabularnewline
\begin{minipage}[t]{0.19\columnwidth}\raggedright\strut
\emph{College Educated}\strut
\end{minipage} & \begin{minipage}[t]{0.46\columnwidth}\raggedright\strut
Dummy for if respondent says s/he has a college diploma\strut
\end{minipage} & \begin{minipage}[t]{0.22\columnwidth}\centering\strut
GB, LB, WVS\strut
\end{minipage}\tabularnewline
\begin{minipage}[t]{0.19\columnwidth}\raggedright\strut
\emph{Unemployed}\strut
\end{minipage} & \begin{minipage}[t]{0.46\columnwidth}\raggedright\strut
Dummy for if respondent is currently unemployed\strut
\end{minipage} & \begin{minipage}[t]{0.22\columnwidth}\centering\strut
GB, LB, WVS\strut
\end{minipage}\tabularnewline
\begin{minipage}[t]{0.19\columnwidth}\raggedright\strut
\emph{Personal Economic Situation}\strut
\end{minipage} & \begin{minipage}[t]{0.46\columnwidth}\raggedright\strut
Overall socioeconomic situation of the respondent. GB: respondent's
personal economic situation (5-item). LB: respondent's socioeconomic
class (5-item). WVS: respondent's financial satisfaction
(10-item).\strut
\end{minipage} & \begin{minipage}[t]{0.22\columnwidth}\centering\strut
GB, LB, WVS\strut
\end{minipage}\tabularnewline
\begin{minipage}[t]{0.19\columnwidth}\raggedright\strut
\emph{Trust Most People}\strut
\end{minipage} & \begin{minipage}[t]{0.46\columnwidth}\raggedright\strut
Dummy for if respondent believes most people can be trusted, 0 if ``can
never be too sure.''\strut
\end{minipage} & \begin{minipage}[t]{0.22\columnwidth}\centering\strut
GB, LB, WVS\strut
\end{minipage}\tabularnewline
\begin{minipage}[t]{0.19\columnwidth}\raggedright\strut
\emph{Targeted Territorial MID}\strut
\end{minipage} & \begin{minipage}[t]{0.46\columnwidth}\raggedright\strut
Dummy if country was targeted in a territorial MID in five years before
survey.\strut
\end{minipage} & \begin{minipage}[t]{0.22\columnwidth}\centering\strut
Palmer et al. (2015)\strut
\end{minipage}\tabularnewline
\begin{minipage}[t]{0.19\columnwidth}\raggedright\strut
\emph{Initiated Territorial MID}\strut
\end{minipage} & \begin{minipage}[t]{0.46\columnwidth}\raggedright\strut
Dummy if country initiated a territorial MID in five years before
survey.\strut
\end{minipage} & \begin{minipage}[t]{0.22\columnwidth}\centering\strut
Palmer et al. (2015)\strut
\end{minipage}\tabularnewline
\begin{minipage}[t]{0.19\columnwidth}\raggedright\strut
\emph{Targeted Non-Territorial MID}\strut
\end{minipage} & \begin{minipage}[t]{0.46\columnwidth}\raggedright\strut
Dummy if country was targeted in a non-territorial MID in five years
before survey.\strut
\end{minipage} & \begin{minipage}[t]{0.22\columnwidth}\centering\strut
Palmer et al. (2015)\strut
\end{minipage}\tabularnewline
\begin{minipage}[t]{0.19\columnwidth}\raggedright\strut
\emph{Initiated Non-Territorial MID}\strut
\end{minipage} & \begin{minipage}[t]{0.46\columnwidth}\raggedright\strut
Dummy if country initiated a non-territorial MID in five years before
survey.\strut
\end{minipage} & \begin{minipage}[t]{0.22\columnwidth}\centering\strut
Palmer et al. (2015)\strut
\end{minipage}\tabularnewline
\begin{minipage}[t]{0.19\columnwidth}\raggedright\strut
\emph{Economic Threat Index}\strut
\end{minipage} & \begin{minipage}[t]{0.46\columnwidth}\raggedright\strut
Interval-level estimate of a country's latent economic ``threat'' from
indicators of GDP per capita, GDP contraction, unemployment, and
consumer price index (c.f. Miller 2017a)\strut
\end{minipage} & \begin{minipage}[t]{0.22\columnwidth}\centering\strut
World Bank\strut
\end{minipage}\tabularnewline
\begin{minipage}[t]{0.19\columnwidth}\raggedright\strut
\emph{State Fragility Index}\strut
\end{minipage} & \begin{minipage}[t]{0.46\columnwidth}\raggedright\strut
Index based on legitimacy and effectiveness on four dimensions:
economic, political, security, and social.\strut
\end{minipage} & \begin{minipage}[t]{0.22\columnwidth}\centering\strut
Marshall and Elzinga-Marshall (2017)\strut
\end{minipage}\tabularnewline
\begin{minipage}[t]{0.19\columnwidth}\raggedright\strut
\emph{Level of Democracy}\strut
\end{minipage} & \begin{minipage}[t]{0.46\columnwidth}\raggedright\strut
Graded response model of ``latent'' level of democracy\strut
\end{minipage} & \begin{minipage}[t]{0.22\columnwidth}\centering\strut
Pemstein, Meserve, and Melton (2010)\strut
\end{minipage}\tabularnewline
\begin{minipage}[t]{0.19\columnwidth}\raggedright\strut
\emph{\% of Parliament Held by Women}\strut
\end{minipage} & \begin{minipage}[t]{0.46\columnwidth}\raggedright\strut
Percent of seats in the country's parliament held by women\strut
\end{minipage} & \begin{minipage}[t]{0.22\columnwidth}\centering\strut
Inter-Parliamentary Union\strut
\end{minipage}\tabularnewline
\begin{minipage}[t]{0.19\columnwidth}\raggedright\strut
\emph{Ethnic Fractionalization}\strut
\end{minipage} & \begin{minipage}[t]{0.46\columnwidth}\raggedright\strut
Probability two randomly selected individuals in a country would belong
to different ethnic groups\strut
\end{minipage} & \begin{minipage}[t]{0.22\columnwidth}\centering\strut
Wimmer, Cederman, and Min (2009)\strut
\end{minipage}\tabularnewline
\begin{minipage}[t]{0.19\columnwidth}\raggedright\strut
\emph{Government Expenditure (\% of GDP)}\strut
\end{minipage} & \begin{minipage}[t]{0.46\columnwidth}\raggedright\strut
Government current consumption expenditures, exluding the military (as
\% of GDP)\strut
\end{minipage} & \begin{minipage}[t]{0.22\columnwidth}\centering\strut
World Bank\strut
\end{minipage}\tabularnewline
\bottomrule
\end{longtable}

\subsubsection{Micro-level}\label{micro-level}

Scholarship on perception and tolerance toward corruption do not specify
elaborate models at the micro-level though demographic variables are
easy to include. Gatti, Paternostro, and Rigolini (2003) find several
demographic variables that are associated with aversion to corruption.
Older citizens are more averse to corruption than younger citizens. I
include the interval age measure from in three of the four models.

Swamy et al. (2001) has a well-traveled argument that women are more
averse to corruption, for which I estimate with a dummy variable if the
respondent self-identifies as a woman. Education generally correlates
negatively with tolerance of corrupt behavior. To test this, I include a
dummy variable that equals 1 if the respondent says s/he has completed a
college education. Those who hold jobs are more likely to be averse to
corruption, suggesting that becoming unemployed is a contextual effect
that would lead individuals to tolerate corruption. I code respondents
who list their employment status as temporarily being without work as
unemployed.

Wealthier citizens whose personal economic situation is more secure are
less inclined to tolerate corruption. However, a direct measure that is
consistent across multiple cross-national public opinion data sets is
simply unavailable. Different survey data sets will ask about a
respondent's personal economic situation in different ways, which leads
to different measures of a respondent's personal economic situation
across the different data sets in my analysis. In the first model using
Globalbarometer data, I use a five-point ordinal measure of an
individual's evaluation of their own economic situation. Those who
respond with ``so so'' are the baseline. This recoded variable ranges
from 2 to -2, where a 2 is a response of ``very good'' and a -2 is a
response of ``very bad''. The Latinobarómetro analyses uses an almost
identical five-point ordinal measure that I recode in a similar manner.
The World Values Survey analysis uses the ten-point financial
satisfaction variable with higher values indicating an individual's
greater satisfaction with his or her financial situation.

Finally, I include a variable relevant to the literature on trust and
corruption. Lack of trust can foster corrupt behavior (which would be
viewed as tolerable) while corrupt behavior can also erode trust. The
variable included is whether the respondent believes most people can be
trusted.

\subsubsection{Macro-level}\label{macro-level}

I also include several country-level predictors as well. We know that
democratic countries are less likely to be corrupt relative to other
regime types. The built-in assumption ehere is democracies empower
citizens to identify and punish instances of corruption, both in
government and society. I include two variables relevant to this
argument. First, I incorporate the Unified Democracies Score (UDS)
democracy estimate for each country in the year prior to the survey year
(Pemstein, Meserve, and Melton 2010). In addition, I also test the
corollary assumption that democracies are less likely to have corruption
because they have mobilized and free press corps that can identify and
punish instances of corruption (Brunetti and Weder 2003). Therefore, I
include Freedom of the Press data from Freedom House for each country in
the survey year. I invert this variable, such that higher values
indicate more, not less, press freedom.

I also control for the general macroeconomic situation and domestic
security of the state. My economic threat index consists of four
macroeconomic indicators: national unemployment rates, the consumer
price index, real GDP per capita, and real GDP per capita change in the
five years prior to the survey year (c.f. Miller 2017a), for which data
come from the World Bank. All four variables are standardized and
appropriately scaled, such that higher values in the index coincide with
worsening economic conditions. My societal threat indicator is the state
fragility index (Marshall and Elzinga-Marshall 2017), which Goel and
Saunoris (2016) found is positively associated with country-level
corruption. The state fragility index measures the legitimacy and
effectiveness of a state on four dimensions (economic, political,
security, and social). Higher values indicate more fragility.

I test the findings from Swamy et al. (2001), who demonstrate that
greater parliamentary representation for women reduces corruption in the
country. The assumption is that the more women are involved in the
national government, the more they are able to foster intolerance toward
corruption among the citizenry. The data come from the
Inter-Parliamentary Union.

Finally, I include two additional variables that Goel and Saunoris
(2016) found coincide with corruption at the country-level. I include
the general government current consumption expenditures (excluding the
military) as a percentage of GDP as it appears in the World Bank data.
This measure of government involvement in the economy generally
coincides with decreasing corruption at the state-level in the Goel and
Saunoris (2016) analysis. Likewise, I include the ethnic
fractionalization measure as it appears in the Ethnic Power Relations
data set (Wimmer, Cederman, and Min 2009). Formally, this measure is the
probability that two randomly selected individuals from a state will
belong to different ethnic groups.

\subsection{Model Choice and Notes}\label{model-choice-and-notes}

Citizens in one country are going to be more similar to their
compatriots than with citizens in other countries. In short,
Argentinians are more like each other than Guatemalans, and so on. Thus
natural nesting of the data, of citizens in countries, violates the
assumption of independence of observations built into standard
statistical models. Therefore, I estimate the following models within
the mixed effects modeling framework as logistic models. Formally, the
estimated equation is

\begin{eqnarray}
Pr(y_{i} = 1)  &=& logit^{-1}(\beta^{0} + \beta^{microlevel_{ij}}*X^{microlevel_{ij}} +  \nonumber \\
               &=&  \beta^{macrolevel_{j}}*X^{macrolevel_{j}} + \alpha{j[i]})  \nonumber \\
\alpha{j[i]}   &=& N(0, \sigma^{2}_{j})              
\end{eqnarray}

where individuals (\emph{i}) are nested within countries (\emph{j}),
\emph{microlevel} refers to the matrix of covariates and the associated
betas of the micro-level controls from the survey data,
\emph{macrolevel} refers to the matrix of covariates and the associated
betas of the macro-level variables that serve as contextual influences
of these individual attitudes, and the random effects are normally
distributed with a mean of zero and a variance to be estimated from the
data.

Finally, I scale all non-binary inputs by two standard deviations to
provide a rough comparability of estimates in the following models (see:
Gelman 2008).

\section{Results}\label{results}

Figure 1 is a dot-and-whisker plot that summarizes the results of of my
analyses of Globalbarometer (GB), Latinobarómetro (LB), and World Values
Survey (WVS) data. I discuss the results below, starting with the
results of the micro-level predictors.

\begin{figure}[htbp]
\centering
\includegraphics{etttc_files/figure-latex/reggraph-1.pdf}
\caption{Mixed Effects Models of Attitudes About Corruption}
\end{figure}

None of the micro-level control variables had consistent effects across
all four models. For example, college-educated respondents were more
likely than those who did not attain a college degree to say corruption
is the country's most important problem and are less likely to view
bribe-taking as justifiable in any way. However, college-educated
respondents were more likely to tolerate government corruption provided
the country's problems were being solved. Older respondents were less
likely to think corruption is the country's most important problem and
were less likely to think bribe-taking is justifiable. However, they are
more tolerant of government corruption in Model 2. Those who believe
most people can be trusted are less likely to tolerate government
corruption but were more likely to think taking a bribe is justifiable.
The variables for personal economic situation were positive and
significant in Model 1 and Model 2 but Model 3 and Model 4 suggest there
is no difference in attitudes about the justifiability of taking a bribe
for those whose personal economic situation is better/more secure
relative to those whose personal economic situation is not as good or
secure.

The macro-level predictors also yield some interesting results. It is
unsurprising that a respondent in the Globalbarometer sample is unlikely
to think of corruption as the country's most important problem when the
country is under considerable economic duress. The respondent is more
likely to select a problem from Globalbarometer's list that deals with
the economy (e.g. ``economics'' or ``poverty''). However, there was no
statistically significant effect of economic threat on questions
pertaining to corruption tolerance in two of the other three models. The
results for the effect of state fragility index in Model 2 and Model 3
are interesting. Citizens in more fragile Latin American states were
less likely to tolerate government corruption but more likely to
tolerate private corruption.

The results for the other macro-level predictors were not as consistent
across the four models. Higher government expenditure as a percentage of
GDP and increasing democracy lead citizens to be less likely to tolerate
government corruption in Model 2 and makes them less likely to tolerate
private corruption in Model 4. There were no discernible effects in
Model 1 and Model 3. Ethnic fractionalization increases the likelihood
of a citizen tolerating government corruption in Model 2. There is
interestingly little support for the Swamy et al. (2001) hypothesis
about women and corruption at the macro-level. Greater parliamentary
representation for women leads to a decrease in the likelihood of
tolerating government corruption but it leads to an increase in the
justifiability of taking a bribe in Model 4. There were no discernible
effects of parliamentary representation for women in Model 1 and Model
3.

However, the variables of interest are the threat variables. I
hypothesize that external threat to territory, as captured by the
presence of targeted territorial dispute in the five years prior to the
survey year, leads citizens to be permissive or tolerant of government
corruption but intolerant of private corruption. Model 1 provides some
prima facie evidence for this. Respondents in states that were targeted
by a militarized threat to revise the state's territorial status quo
were unlikely to see corruption as the country's most important problem.
Consistent with my argument, the salient nature of territorial threat
shifts focus of national problems from corrupt behavior toward national
security. The result for the territorial threat variable from Model 2
builds on the finding from Model 1. Citizens in Latin America under
territorial threat, such as those in Ecuador, Peru, and Central American
states like Nicaragua involved in the Gulf of Fonseca dispute, are more
likely to tolerate government corruption. However, Model 3 shows that
those same citizens who tolerate government corruption are unlikely to
tolerate private corruption. The coefficient for the targeted
territorial MID variable that was positive and statistically significant
in Model 2 is negative and statistically significant in Model 3. The
findings from Model 3 regarding the effect of territorial threat on
tolerance toward private corruption are supported in Model 4 using World
Values Survey data. Being targeted in a territorial dispute is unique
among the conflict indicators for producing this asymmetry in attitudes
toward the tolerance of corruption.

After estimating the four models in Figure 1, I generated first
differences in predicted probabilities as quantities of interest (see
King, Tomz, and Wittenberg 2000). The quantities provided in Figure 2
are first differences in the predicted probabilities of a value of
\emph{y} in an increase from 0 to 1 in the targeted territorial MID
variable. All other values in these simulations are held at the typical
value.

\begin{figure}[htbp]
\centering
\includegraphics{etttc_files/figure-latex/sims-1.pdf}
\caption{Simulated Quantities of Interest From the Regression Models}
\end{figure}

The first difference in the predicted probability of a respondent in the
Globalbarometer sample labeling corruption as the country's most
important problem is -.033. Whereas the predicted probability of naming
corruption as the country's most important problem is .071 in the
absence of a territorial threat, a first difference of -.033 corresponds
with over a negative 53\% change in the predicted probability.
Consistent with my argument, the presence of an external territorial
threat leads citizens to value their security over monitoring the
government's activities for corrupt behavior.

The first difference in the simulated predicted probability of a
respondent tolerating government corruption in the Latinobarómetro
analysis is .103. Whereas the simulated probability of tolerating
government corruption provided the country's problems were being solved
was .59 in the data, this increase of .103 coincides with a 17.5\%
change in the predicted probability.

The effects are substantively large in Model 3 and Model 4 as well. The
first difference in the Latinobarómetro analysis on the justifiability
of taking a bribe was -.120, which constitutes a negative 54\% change in
the probability of thinking a bribe is justifiable when the citizen's
state is under territorial threat. In the fourth model using World
Values Survey data, this percentage change for a first difference of
-.107 is negative 35\%. All told, Figure 2 provides quantities of
interest underscoring the effect of territorial threat on attitudes
toward corruption observed in Figure 1.

\section{Conclusion}\label{conclusion}

This manuscript started with two interesting questions that often go
unasked in the corruption literature. First, when do citizens tolerate
corruption by their national government? Though kleptocracy, graft, and
other common forms of corrupt behavior seem inimical to individual and
societal welfare and are often punished upon the perception of these
activities being present, the perception of corruption and the
\emph{tolerance} of corruption are separate concepts. Second, can the
same behavior that is condemned as corrupt in society be condonable
behavior for government officials? In short, can individuals tolerate
government corruption but condemn private corruption?

Answers proposed in this manuscript draw attention to the regional
threat environment. Citizens under a salient, external threat, like a
territorial threat, will condone government corruption. Citizens who
value their security are likely to loosen oversight on other government
activities. This allows for government officials to behave corruptly
with respect to the state's resources under the condition that the
government is providing for the citizens' security. Behavior that would
otherwise be condemned in peace time will be tolerated under times of
threat. However, this tolerance of government corruption does not extend
to corruption in society. Private citizens who behave corruptly in
society are seen as undermining the common good through seeking personal
gains in a critical time of threat. Citizens under territorial threat
will condemn corruption in society as a social behavior but will condone
it as a political behavior for their government.

This manuscript suggests rethinking how scholars gauge individual-level
attitudes toward corruption. Not all forms of corruption are the same
nor are all attitudes toward corruption equivalent. Individuals can
observe corruption and tolerate its presence and can extend this
tolerance to some people and not others. However, our most common way of
approaching this issue of corruption tolerance is the standard ten-point
survey item on the justifiability of bribe-taking given by the World
Values Survey. This gives only a cursory probe of attitudes toward
corruption. Latinobarómetro provides a unique data source to test the
full implications of my argument. Future installments of cross-national
survey data should dig deeper on attitudes toward corruption.

Beyond promoting a refinement of the extant political science literature
on corruption, the analyses presented here have some important
implications for the literature on territorial conflict. That
individuals tolerate government corruption but condemn private
corruption conforms well with the Hutchison (2011) argument that
territorial threat allows governments to mobilize and shape public
opinion and behavior in accordance with the government's preferences.
Thus, the government is able to communicate to citizens that it needs
the necessary liberties in order to work toward national defense and to
behave corruptly with respect to other aspects of the budget. This same
cue from the government is likely to induce greater political
participation, as Hutchison does find, because the government
communicates that everyone must do their part in order to achieve the
common good. This process then leads individuals to be intolerant of
corrupt behavior among themselves that can be seen as undermining that
goal.

\section{References}\label{references}

\setlength{\parindent}{-0.2in} \setlength{\leftskip}{0.2in}
\setlength{\parskip}{8pt} \vspace*{-0.2in} \noindent

\hypertarget{refs}{}
\hypertarget{ref-acemoglurobinson2006eodd}{}
Acemoglu, Daron, and James A. Robinson. 2006. \emph{Economic Origins of
Dictatorship and Democracy}. New York, NY: Cambridge University Press.

\hypertarget{ref-argandona2003ppc}{}
Argandona, Antonio. 2003. ``Private-to-Private Corruption.''
\emph{Journal of Business Ethics} 47 (3): 253--67.

\hypertarget{ref-beckmaher1986cbb}{}
Beck, Paul J., and Michael W. Maher. 1986. ``A Comparison of Bribery and
Bidding in Thin Markets.'' \emph{Economics Letters} 20 (1): 1--5.

\hypertarget{ref-boix2003dr}{}
Boix, Carles. 2003. \emph{Democracy and Redistribution}. New York, NY:
Cambridge University Press.

\hypertarget{ref-brunettiweder2003fpbn}{}
Brunetti, Aymo, and Beatrice Weder. 2003. ``A Free Press Is Bad News for
Corruption.'' \emph{Journal of Public Economics} 87: 1801--24.

\hypertarget{ref-bdmetal2003lps}{}
Bueno de Mesquita, Bruce, Alastair Smith, Randolph M. Siverson, and
James D. Morrow. 2003. \emph{The Logic of Political Survival}.
Cambridge, MA: MIT Press.

\hypertarget{ref-bdm2007psp}{}
Bueno de Mesquita, Ethan. 2007. ``Politics and the Suboptimal Provision
of Counterterror.'' \emph{International Organization} 61 (1): 9--36.

\hypertarget{ref-canacheallison2005ppc}{}
Canache, Damarys, and Michael E. Allison. 2005. ``Perceptions of
Political Corruption in Latin American Democracies.'' \emph{Latin
American Politics and Society} 47 (3): 91--111.

\hypertarget{ref-caputo2005dla}{}
Caputo, Dante. 2005. \emph{Democracy in Latin America: Toward a
Citizens' Democracy}. Edited by Richard Jones. Buenos Aires: Aguilar,
Altea, Taurus, Alfaguara, S.A.

\hypertarget{ref-changchu2006ct}{}
Chang, Eric C.C., and Yun-han Chu. 2006. ``Corruption and Trust:
Exceptionalism in Asian Democracies?'' \emph{Journal of Politics} 68
(2): 259--71.

\hypertarget{ref-changkerr2009dvh}{}
Chang, Eric C.C., and Nicholas N. Kerr. 2009. ``Do Voters Have Different
Attitudes Toward Corruption? The Sources and Implications of Popular
Perceptions and Tolerance of Political Corruption.'' \emph{Afrobarometer
Working Paper No. 116}.

\hypertarget{ref-dellaporta2000scbg}{}
Della Porta, Donatella. 2000. ``Social Capital, Beliefs in Government,
and Political Corruption.'' In \emph{Disaffected Democracies: What's
Troubling the Trilateral Countries}, 202--29. Princeton, NJ: Princeton
University Press.

\hypertarget{ref-dongaetal2012cc}{}
Donga, Bin, Uwe Dulleckb, and Benno Torgler. 2012. ``Conditional
Corruption.'' \emph{Journal of Economic Psychology} 33 (3): 609--27.

\hypertarget{ref-gattietal2003iatc}{}
Gatti, Roberta, Stefano Paternostro, and Jamele Rigolini. 2003.
``Individual Attitudes Toward Corruption: Do Social Effects Matter?''
\emph{World Bank Policy Research Working Paper} No. 3122.

\hypertarget{ref-gelman2008sri}{}
Gelman, Andrew. 2008. ``Scaling Regression Inputs by Dividing by Two
Standard Deviations.'' \emph{Statistics in Medicine} 27 (15): 2865--73.

\hypertarget{ref-gibler2010oi}{}
Gibler, Douglas M. 2010. ``Outside-in: The Effects of Territorial Threat
on State Centralization.'' \emph{Journal of Conflict Resolution} 54 (4):
519--42.

\hypertarget{ref-gibler2012tp}{}
---------. 2012. \emph{The Territorial Peace: Borders, State
Development, and International Conflict}. New York, NY: Cambridge
University Press.

\hypertarget{ref-gibleretal2016amid}{}
Gibler, Douglas M., Steven V. Miller, and Erin K. Little. 2016. ``An
Analysis of the Militarized Interstate Dispute (MID) Dataset,
1816-2001.'' \emph{International Studies Quarterly} 60 (4): 719--30.

\hypertarget{ref-goelsaunoris2016mbed}{}
Goel, Rajeev K., and James W. Saunoris. 2016. ``Military Buildups,
Economic Development, and Corruption.'' \emph{The Manchester School} 84
(6): 697--722.

\hypertarget{ref-goeletal2015psb}{}
Goel, Rajeev K., Jelena Budak, and Edo Rajh. 2015. ``Private Sector
Bribery and Effectiveness of Anti-Corruption Policies.'' \emph{Applied
Economic Letters} 22 (10): 759--66.

\hypertarget{ref-heidenheimer2002ppc}{}
Heidenheimer, Arnold J. 2002. ``Perspectives on the Perception of
Corruption.'' In \emph{Political Corruption: Concepts and Contexts},
141--54. New Brunswick, NJ: Transaction Publishers.

\hypertarget{ref-huntington1968pocs}{}
Huntington, Samuel P. 1968. \emph{Political Order in Changing
Societies}. New Haven, CT: Yale University Press.

\hypertarget{ref-hutchison2011ttm}{}
Hutchison, Marc L. 2011. ``Territorial Threats, Mobilization, and
Political Participation in Africa.'' \emph{Conflict Management and Peace
Science} 28 (3): 183--208.

\hypertarget{ref-hutchisongibler2007ptt}{}
Hutchison, Marc L., and Douglas M. Gibler. 2007. ``Political Tolerance
and Territorial Threat: A Cross-National Study.'' \emph{Journal of
Politics} 69 (1): 128--42.

\hypertarget{ref-katzaetal1994frmd}{}
Katza, Roger C., Jennifer Santmana, and Pamela Loneroa. 1994. ``Findings
on the Revised Morally Debatable Behaviors Scale.'' \emph{Journal of
Psychology} 128 (1): 15--21.

\hypertarget{ref-kingetal2000mmsa}{}
King, Gary, Michael Tomz, and Jason Wittenberg. 2000. ``Making the Most
of Statistical Analyses: Improving Interpretation and Presentation.''
\emph{American Journal of Political Science} 44 (2). Midwest Political
Science Association: 347--61.

\hypertarget{ref-knack2007mc}{}
Knack, Stephen. 2007. ``Measuring Corruption: A Critique of Indicators
in Eastern Europe and Central Asia.'' \emph{Journal of Public Policy} 27
(3): 255--91.

\hypertarget{ref-kramer2005agw}{}
Kramer, Andrew E. 2005. ``Amid Growing Wealth, Nepotism and Nationalism
in Kazakhstan.'' \emph{New York Times} December 23.

\hypertarget{ref-krueger1974pers}{}
Krueger, Anne. 1974. ``The Political Economy of the Rent-Seeking
Society.'' \emph{American Economic Review} 64: 291--303.

\hypertarget{ref-leys1989pch}{}
Leys, Colin. 1989. ``What Is the Problem About Corruption?'' In
\emph{Political Corruption: A Handbook}, edited by Arnold J.
Heidenheimer, Michael Johnston, and Victor T. LeVine. New Brunswick, NJ:
Transaction Publishers.

\hypertarget{ref-lillis2010ckh}{}
Lillis, Joanna. 2010. ``Citizens in Kazakhstan Are High on Nazarbayev,
Tepid on Democratization.'' \emph{EurasiaNet} May 26.
\url{http://www.eurasianet.org/node/61158}.

\hypertarget{ref-manzettiwilson2007wdcg}{}
Manzetti, Luigi, and Carole J. Wilson. 2007. ``Why Do Corrupt
Governments Maintain Public Support?'' \emph{Comparative Political
Studies} 40 (8): 949--70.

\hypertarget{ref-marshallmarshall2017gr}{}
Marshall, Monty G., and Gabrielle Elzinga-Marshall. 2017. \emph{Global
Report 2017: Conflict, Governance and State Fragility}. Vienna, VA:
Center for Systemic Peace.

\hypertarget{ref-mauro1995cg}{}
Mauro, Paolo. 1995. ``Corruption and Growth.'' \emph{Quarterly Journal
of Economics} 110 (3): 681--712.

\hypertarget{ref-miller2013tdpi}{}
Miller, Steven V. 2013. ``Territorial Disputes and the Politics of
Individual Well-Being.'' \emph{Journal of Peace Research} 50 (6):
677--90.

\hypertarget{ref-miller2017etst}{}
---------. 2017a. ``Economic Threats or Societal Turmoil? Understanding
Preferences for Authoritarian Political Systems.'' \emph{Political
Behavior} 39 (2): 457--78.

\hypertarget{ref-miller2017ieea}{}
---------. 2017b. ``Individual-Level Expectations of Executive Authority
Under Territorial Threat.'' \emph{Conflict Management and Peace Science}
34 (5): 526--45.

\hypertarget{ref-mishlerrose2001wopt}{}
Mishler, William, and Richard Rose. 2001. ``What Are the Origins of
Political Trust? Tresting Institutional and Cultural Theories in
Post-Communist Societies.'' \emph{Comparative Political Studies} 34 (1):
30--62.

\hypertarget{ref-morrisklesner2007ct}{}
Morris, Stephen D., and Joseph L. Klesner. 2007. ``Corruption and Trust:
Theoretical Considerations and Evidence from Mexico.'' \emph{Comparative
Political Studies} 43 (10): 1258--85.

\hypertarget{ref-olken2009cpcr}{}
Olken, Benjamin A. 2009. ``Corruption Perceptions Vs. Corruption
Reality.'' \emph{Journal of Public Economics} 93 (7-8): 950--64.

\hypertarget{ref-palmeretal2015mid4}{}
Palmer, Glenn, Vito D'Orazio, Michael Kenwick, and Matthew Lane. 2015.
``The Mid4 Dataset, 2002--2010: Procedures, Coding Rules and
Description.'' \emph{Conflict Management and Peace Science} 32 (2):
222--42.

\hypertarget{ref-pani2011hnv}{}
Pani, Marco. 2011. ``Hold Your Nose and Vote: Corruption and Public
Decisions in a Representative Democracy.'' \emph{Public Choice} 148:
163--96.

\hypertarget{ref-pemsteinetal2010dc}{}
Pemstein, Daniel, Stephen A. Meserve, and James Melton. 2010.
``Democratic Compromise: A Latent Variable Analysis of Ten Measures of
Regime Type.'' \emph{Political Analysis} 18 (4): 426--49.

\hypertarget{ref-roseetal1998da}{}
Rose, Richard, William Mishler, and Christian Haerpfer. 1998.
\emph{Democracy and Its Alternatives: Understanding Post-Communist
Societies}. Baltimore, MD: The Johns Hopkins University Press.

\hypertarget{ref-roseackerman1999thc}{}
Rose-Ackerman, Susan. 2001. ``Trust, Honesty, and Corruption: Reflection
on the State-Building Process.'' \emph{European Journal of Sociology} 42
(1): 27--71.

\hypertarget{ref-schlesingermeier2002vcas}{}
Schlesinger, Thomas J., and Kenneth J. Meier. 2002. ``Variation in
Corruption Among the American States.'' In \emph{Political Corruption:
Concepts and Contexts}, edited by Arnold J. Heidenheimer and Michael
Johnston, 627--44. New Brunswick, NJ: Transaction Publishers.

\hypertarget{ref-seligson2002icrl}{}
Seligson, Mitchell A. 2002. ``The Impact of Corruption on Regime
Legtimacy: A Comparative Study of Four Latin American Countries.''
\emph{Journal of Politics} 64 (2): 408--33.

\hypertarget{ref-senesevasquez2003uet}{}
Senese, Paul D., and John A. Vasquez. 2003. ``A Unified Explanation of
Territorial Conflict: Testing the Impact of Sampling Bias, 1919-1992.''
\emph{International Studies Quarterly} 47 (2): 275--98.

\hypertarget{ref-sullivanetal1979acpt}{}
Sullivan, John, George Marcus, and James Pierson. 1979. ``An Alternative
Conceptualization of Political Tolerance: Illusory Increases,
1950s-1970s.'' \emph{American Political Science Review} 73 (3): 781--94.

\hypertarget{ref-swamyetal2001gc}{}
Swamy, Anand, Stephen Knack, Young Lee, and Omar Azfar. 2001. ``Gender
and Corruption.'' \emph{Journal of Development Economics} 64: 25--55.

\hypertarget{ref-tir2010td}{}
Tir, Jaroslav. 2010. ``Territorial Diversion: Diversionary Theory of War
and Territorial Conflict.'' \emph{Journal of Politics} 72 (2): 413--25.

\hypertarget{ref-vasquez2009twp}{}
Vasquez, John A. 2009. \emph{The War Puzzle Revisited}. New York, NY:
Cambridge University Press.

\hypertarget{ref-williams1999nco}{}
Williams, Robert. 1999. ``New Concepts for Old?'' \emph{Third World
Quarterly} 20 (3): 503--13.

\hypertarget{ref-wimmeretal2009epac}{}
Wimmer, Andreas, Lars-Erick Cederman, and Brian Min. 2009. ``Ethnic
Politics and Armed Conflict: A Configurational Analysis of a New Global
Data Set.'' \emph{American Sociological Review} 74 (2): 316--37.




\newpage
\singlespacing 
\end{document}
